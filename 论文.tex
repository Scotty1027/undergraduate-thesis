% 正文的中文、符号采用宋体小四,阿拉伯数字及西文字母、符号采用Times New Roman,小四。
\documentclass[UTF8,zihao=-4]{ctexart}
\setCJKmainfont{SimSun}
\setmainfont{Times New Roman}
\usepackage{geometry}
\usepackage{setspace}
\usepackage{graphicx}
\usepackage{zhlipsum}
\usepackage{lipsum}
\usepackage{fancyhdr}
\usepackage{pdfpages}
\pagestyle{fancy}
% 设计(论文)除封皮外,各页均应加页眉,单线上居中打印页眉文字,页眉文字均为宋体五号“长春理工大学本科毕业设计”或“长春理工大学本科毕业论文”。
\chead{长春理工大学本科毕业论文}
\lhead{}
\rhead{}
% 设计(论文)页码:宋体,小五,页面底端居中
\cfoot{\songti\zihao{-5}\thepage}
\geometry{a4paper,left=3.17cm,right=3.17cm,top=2.54cm,bottom=2.54cm}  %上下边距2.54cm,左(+装订线0cm)右边距3.17cm,
\title{基于朴素贝叶斯的垃圾邮件过滤算法}
\author{150521310 何程斌}
\date{}
\renewcommand{\abstractname}{}
\ctexset{
  % 各章一级标题序号及标题:宋体三号加粗居中
  % 目录标题貌似也是 section 标题格式控制
  % 目录题头为三号宋体字加粗居中书写,目录中各章题序及标题均用宋体,小四;多倍行距1.25,页码对齐,目录自动生成。
  section = {
    name = {第,章},
    format+=\songti\zihao{3}\textbf\centering
  },
  % 各章二级标题序号及标题:宋体四号加粗,缩进2个字符;
  subsection = {
    format+=\songti\zihao{4}\textbf
    % indent=2\ccwd
  },
  % 各章三级标题序号及标题(含三级标题以下标题):宋体小四加粗,缩进2个字符。
  subsubsection = {
    format+=\songti\zihao{-4}\textbf
    % indent=2\ccwd
  }
}
\begin{document}
% 封面页
\includepdf[page=1-2]{论文封面.pdf}

% 摘要、目录等正文之前的页码用罗马数字(I、II、III……),且分别编码
% 设计(论文)的中文和外文摘要置于目录前,并编入目录。
\pagenumbering{Roman}
\begin{abstract}
  \addcontentsline{toc}{section}{摘要}
  % 中文摘要题头选用模板中样式所定义的“标题1”,居中;宋体,居中,三号,加粗;多倍行距1.25,段前0行,段后1行,取消网格对齐。
  \begin{center}
    {\noindent{} \linespread{1.25}\songti\zihao{3}\textbf{摘要}}
  \end{center}
  \addvspace{1\baselineskip}

  % 摘要正文选用模板中样式所定义的“正文”,段落首行缩进2个字符,宋体,小四;多倍行距1.25,段前、段后均为0行,取消网格对齐。
  \linespread{1.25}\songti\zihao{-4}\zhlipsum*[1]\\

  % 关键词与摘要之间空一行。中文关键词之间用2个空格间隔,黑体小四顶格书写。
  \noindent\heiti\zihao{-4}\textbf{关键字:}摘要  \LaTeX  中文
  % \end{spacing}
\end{abstract}
\newpage

\begin{abstract}
  \addcontentsline{toc}{section}{Abstract}
  % 外文摘要题头“Abstract”不可省略。标题“Abstract”选用模板中的样式所定义的“标题1”,居中;Times New Roman,居中,三号,加粗;多倍行距1.25,段前0行,段后1行,取消网格对齐。
  \begin{center}
    {\noindent{} \linespread{1.25}\zihao{3}\textbf{Abstract}}
  \end{center}
  \addvspace{1\baselineskip}

  % Abstract正文设置:段落首行缩进2字;Times New Roman,小四;多倍行距1.25,段前、段后均为0行,取消网格对齐。
  \linespread{1.25}\zihao{-4} \lipsum*[1]\\

  % Key words与Abstract正文之间空一行。Key words之间用分号间隔,Times New Roman,小四,顶格书写。
  \noindent\zihao{-4}\textbf{Key Words:}摘要  \LaTeX  中文
  % \end{spacing}
\end{abstract}
\newpage

% TOC 页
\linespread{1.25}
\tableofcontents
\newpage

% 正文以后的页码用阿拉伯数字(1、2、3……)编排页码
\pagenumbering{arabic}
\section{引言}
\zhlipsum*[2]

\section{国内外研究现状}

\subsection{研究现状一}
\zhlipsum*[3]

\subsubsection{研究现状二}
\zhlipsum*[3]

\section{总体设计}
\zhlipsum*[4]

\section{详细设计}
\zhlipsum*[5]

\section{结论}
\zhlipsum*[6]


\begin{thebibliography}{99}
  % 参考文献正文字体:宋体五号,多倍行距1.25,段前、段后均为0行。
  \linespread{1.25}\songti\zihao{5}
  \bibitem{ref1}Russell, Stuart; Norvig, Peter. Artificial Intelligence: A Modern Approach 2nd. Prentice Hall. 2003 [1995]. ISBN 978-0137903955.
  \bibitem{ref2}杨雷,曹翠玲,孙建国,张立国.改进的朴素贝叶斯算法在垃圾邮件过滤中的研究[J].通信学报,2017,38(04):140-148.
  \bibitem{ref3}张培,纪鸿旭,李璐.基于朴素贝叶斯的中文垃圾邮件过滤[J].信息与电脑(理论版),2017(07):79-81.
  \bibitem{ref4}马哲. 垃圾邮件过滤系统的研究与实现[D]. 浙江大学, 2005.
  \bibitem{pkuseg}Xu Sun, Houfeng Wang, Wenjie Li. Fast Online Training with Frequency-Adaptive Learning Rates for Chinese Word Segmentation and New Word Detection. Proceedings of ACL. 253–262. 2012
\end{thebibliography}
\end{document}